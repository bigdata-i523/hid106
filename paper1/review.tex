\documentclass[sigconf]{acmart}

\usepackage{graphicx}
\usepackage{hyperref}
\usepackage{todonotes}

\usepackage{endfloat}
\renewcommand{\efloatseparator}{\mbox{}} % no new page between figures

\usepackage{booktabs} % For formal tables

\settopmatter{printacmref=false} % Removes citation information below abstract
\renewcommand\footnotetextcopyrightpermission[1]{} % removes footnote with conference information in first column
\pagestyle{plain} % removes running headers

\newcommand{\TODO}[1]{\todo[inline]{#1}}

\begin{document}

\author{Qiaoyi Liu}
\affiliation{%
  \institution{Indiana University of Bloomington}
  \streetaddress{3209 E 10th St}
  \city{Bloomington} 
  \state{Indiana} 
  \postcode{47408}
}
\email{ql30@umail.iu.edu}


\title{Big Data Analytics in Groceries Stores}

% The default list of authors is too long for headers}
\renewcommand{\shortauthors}{Q. Liu}

\begin{abstract}
This paper helps us understanding how big data is working 
       in Groceries store and how Big Data helping their business.
\end{abstract}

\keywords{i423, hid106, Data Science, Big Data Analytics, Cloud Computing,customer study}

\maketitle

\section{Introduction}

Today, numerous market chains perform an assessment of their client/customers on a massive set of data, discovering experiences that assist them better includes customers and, thus, drive income. Discerning how to use big data is vital in an industry where profits are razor thin, and waste management is a broad issue. By gathering and evaluating customer data, grocery stores can sharpen their approach to everything from advertising exercises and pricing to product classification and customer’s benefit \cite{2,4}. With the appropriate analytical tool, grocery stores can unite various sources of data and get information progressively in real time, letting them precisely conjecture product demand, improve stock levels and turn-rates, and lessen waste of perishable products. The article will examine the importance of Big Data Analytics in Groceries stores, its relevancy, its use as a competitive advantage tool to attracting customers, in addition to determining customer demand.  Grocer Loyalty program databases, rich with a point to point customer information, have been in presence for quite a long time, giving food merchants a clear preferred standpoint (for those that have utilized this information) contrasted with different retailers \cite{1}. Grocers have had a head start beginning on using this information in better understanding shopper behaviors and shopping preference. Nonetheless, with the coming of new technological innovations, new contenders, new channels and the rise of a 'constantly-on' and 'time-starved' purchaser base with a bunch of advantageous shopping choices – the grocery industry is presently trailing different retailers in the capacity to use these new 'huge' data sources to advance their investigative abilities from interactions to transaction\cite{5}. Specifically, the development of new types of data sources – big data– offers a chance to Small to Medium Size grocer's equal opportunity to compete with big chains of supermarkets. The proceeding section highlights the relevance of the big data.   

\section{RELEVANCY OF BIG DATA ANALYTICS IN GROCERIES STORES}

\subsection{Increases the customer shopping experience }
As per a current SHSFoodThink white paper "Are We Chain Obsessed?" 64{\%} of customers said that the previous shopping experience is what makes them keep coming back—not the items themselves \cite{3}. By utilizing bits of knowledge received from the information transaction database, online networking, promotional activity, customers purchasing behavior, and client movement patterns, grocery stores can find a way to guarantee they are engaged with their customers that matter most.  
	 For instance, they can investigate customers shopping movement to enhance the layout of their store, or recognize attrition risks for clients who have not as of late bought staple things, similar to milk. In like manner, chains can construct item varieties demonstrated with the customer needs and purchase patterns in certain regions \cite{1, 3, 5}. Regardless of whether it is through reconsidering store layout or furnishing store attended with mobile apps to better serve clients, analytics can enable grocers to change consumer’s expectations. 



\subsection{Restructure the Supply Chain}

Grocery stores can likewise utilize analytic to investigate the production of their products, monitor production processes, and quality control, and improve straightforwardness with buyers about their sustenance production practices of foods \cite{2}. Suppliers remain to profit from the evaluation also, with access to secure, customized content of information identified with performance sales of the product, stock, margins, and marketing effectiveness. Giving supplier an opportune profitable business knowledge that supports joint ventures, drives performance, and decreases waste products

\subsection{Build Superior Marketing Programs}

Loyalty programs furnish grocery merchants with an abundance of data to enable them to distinguish client segments and precisely characterize item preferences. By joining this information with different data sources—like healthful patterns, favored technique for accepting marketing promotion, customer movement patterns, and weather-related event—grocery merchants can concentrate on enhancing, and derive income from, the general shopping experience \cite{3}. For instance, grocery retailers can utilize analytics to customize the advancements they offer to clients given what they are well on the way to buy. They can likewise time advancements fittingly, and offer codes to customers who often as possible buy certain things. 

\subsection{Improves HR Strategies}

Supermarket stores utilize analytics to manage work-related decisions. Information freely accessible through online networking accounts and different means can be examined in conjunction with a grocer's internal information to direct decision identified with selection and recruitment, employee termination, and performance management and advancements \cite{4}. For example, an investigation of late action on LinkedIn can reveal insight into which representatives are destined to leave an organization.  
 	Grocery merchants can likewise break down information to control the advancement approaches that will build workforce performance. For example, they could explore different avenues regarding organizing a social gathering for representatives at a subset of their stores, and analyze information on profitability, morale, and turnover in the preceding months \cite{1}. They may find that the gathering information prompted a more positive workplace where workers feel more noteworthy engagement at work, and soon after that, they could roll the strategy out to different stores. 


\subsection{Using big data for competitive advantage and attracting customers}

Numerous grocery stores have been utilizing transaction and client information for a considerable length of time, despite the fact that many still have not completely used all that can be proficient with these types of information. For Small to Medium Sized grocery merchants, many have swung to subcontracted point solutions because of an absence of available analytics assets and potential framework investment required \cite{3, 4}. The issue with point solutions recently is that – they independently work out for a particular business section and the evaluation is cookie cutter. In this way, the 'information' is not coordinated and hard if not difficult to give an all-encompassing picture of client conduct overall touch focuses for instance.  Nor are the investigations offering a cross-functional observation that is pertinent to all business partners as far as driving differentiation in the commercial center in promoting, advertising, store operations and supply chain. 
 	As far as utilizing 'new' data sources, for example, mobile, social and text, the industry is particularly occupied with a discovery' phase of investigation with an assortment of center sections, testing and figuring out how to extricate an incentive from these rich new sources of information. There are two common paths grocery merchants takes with little respect of the 'size' of the organization: to start with is Strategic Commitment, in which there is C-level (hierarchical) commitment making the venture in the assets to get the majority of the in-house data and evaluated it \cite{1}.
 	Presently like never before, information, analytics, and IP are seen as vital resources and competitive discriminators. The other is Business Discovery; in which grocery merchants outsource to an Analytics as a Service firm to use internal and external information. Performing analytics speeds the construction of business advantages creating new users case and helps catch 'quick wins' before making resource commitment to technological innovation and human capital in advance \cite{4}. In view of progress, and a wit, trusted stakeholder willing to share the techniques and explanatory models, can assist grocery merchants to proceed with an outsourced administrations supplier or relocate the data, analytics in addition to IP in-house. 


\section{RECOMMENDATIONS}
\subsection{Real-time insight on product demand}

Nowadays, retailers can get to information on item demand levels instantly on a chain of stores. Nevertheless, numerous merchants are still in the earliest stages in regards to evaluating and monetizing the huge amount accessible data \cite{5}. This prompts stocking deficits, for example, evaluating item demanded based exclusively on past historical information. It can likewise convey about wrong promoting endeavors: If a customer purchased ketchup on Saturday, an email coupon for it on Sunday is not well planned and make little sense to the shopper.  
 	This is the place data from store loyalty programs in addition to credit card sales can prove to be useful. Its data can be utilized to define needs of the customers in future. For example, grocery merchants can use data analytics to decide how regularly customers purchase sugar, flavors, or different items, and after that send every family unit coupons given their propensity to buy \cite{3}.



\subsection{Enhancing in-store stock management }

Perishable basic supplies, for example, dairy, meat, and fish call for precise stock administration, regularly on an hourly premise. Client analytics and prediction tools can enable grocery merchants to calibrate their inventory levels by assessing buyer purchasing behavior and requested products from various viewpoints and situations \cite{3}. 
	For example, grocery retailers might need to screen cycles like when customers go for particular nourishment, purchasing patterns amid sales deals when storing activity peaks or seasonally inspired buys. As indicated by a report from Manthan, this methodology worked for U.K. food grocery merchant Waitrose: a deeper understanding of buyer purchasing behavior and demand outlines using cutting edge client analytics and predicting tools helped the store \cite{4}. Concurrently, retailers can utilize these systems to all the more deftly change their stock levels and amplify high-buy products. 


\subsection{Leveraging Predictive Analytics}

Amazon spearheaded item proposal engine: the "if you purchased that, you may like this" invention. This strategic changing web-based shopping feature mirrors the retailer's profound assessment of buyers' shopping basket.  Proposal engine is intended to enable customers to find items they were not sorting out but rather would be interested in purchasing \cite{1}. Today, general grocery merchants are progressively tapping the global innovation behind proposal engine: predictive analytics. This kind of assessment measures future patterns in light of present and past information, and it can enable stores to improve business. Information is driven, all-encompassing assessment of "purchasing triggers, for example, regularity, weather, stock, and advancements, is progressively informing grocery stores' product blend, marketing plans, and sales forecast [5]. Furnished with these information-driven tools, stores can better distinguish what items customers need today and what they will be demanding in future, and this learning will enable them to stay competitive for a considerable length of time to come.  

\section{Conclusion}

Big data analytics is profound tool assisting grocery merchant establishing insightful information concerning the market structure and sales demand. With the appropriate analytical tool, grocery stores can unite various sources of data and get information progressively in real time, letting them precisely conjecture product demand, improve stock levels and turn-rates, and lessen waste of perishable products. Advancement in information technology is offering new means –Big data analytics that Small to Medium Business such as grocery merchant can use to drive the products sales. Big as discussed previously, assist grocery merchant to increase their customer experience, restructure supply chain, create superior market programs, improve HR strategies, and creates them a competitive advantage.

\bibliographystyle{ACM-Reference-Format}
\bibliography{report} 

\section{Bibtex Issues}
\todo[inline]{Warning--no number and no volume in 5}
\todo[inline]{Warning--page numbers missing in both pages and numpages fields in 5}
\todo[inline]{Warning--page numbers missing in both pages and numpages fields in 1}
\todo[inline]{Warning--page numbers missing in both pages and numpages fields in 2}
\todo[inline]{Warning--can't use both author and editor fields in 3}
\todo[inline]{Warning--can't use both volume and number fields in 3}
\todo[inline]{Warning--empty address in 3}
\todo[inline]{(There were 7 warnings)}
\section{Issues}

\DONE{Example of done item: Once you fix an item, change TODO to DONE}

\subsection{Assignment Submission Issues}

    \TODO{Do not make changes to your paper during grading, when your repository should be frozen.}

\subsection{Uncaught Bibliography Errors}

    \TODO{Missing bibliography file generated by JabRef}
    \TODO{Bibtex labels cannot have any spaces, \_ or \& in it}
    \TODO{Citations in text showing as [?]: this means either your report.bib is not up-to-date or there is a spelling error in the label of the item you want to cite, either in report.bib or in report.tex}

\subsection{Formatting}

    \TODO{Incorrect number of keywords or HID and i523 not included in the keywords}
    \TODO{Other formatting issues}

\subsection{Writing Errors}

    \TODO{Errors in title, e.g. capitalization}
    \TODO{Spelling errors}
    \TODO{Are you using {\em a} and {\em the} properly?}
    \TODO{Do not use phrases such as {\em shown in the Figure below}. Instead, use {\em as shown in Figure 3}, when referring to the 3rd figure}
    \TODO{Do not use the word {\em I} instead use {\em we} even if you are the sole author}
    \TODO{Do not use the phrase {\em In this paper/report we show} instead use {\em We show}. It is not important if this is a paper or a report and does not need to be mentioned}
    \TODO{If you want to say {\em and} do not use {\em \&} but use the word {\em and}}
    \TODO{Use a space after . , : }
    \TODO{When using a section command, the section title is not written in all-caps as format does this for you}\begin{verbatim}\section{Introduction} and NOT \section{INTRODUCTION} \end{verbatim}

\subsection{Citation Issues and Plagiarism}

    \TODO{It is your responsibility to make sure no plagiarism occurs. The instructions and resources were given in the class}
    \TODO{Claims made without citations provided}
    \TODO{Need to paraphrase long quotations (whole sentences or longer)}
    \TODO{Need to quote directly cited material}

\subsection{Character Errors}

    \TODO{Erroneous use of quotation marks, i.e. use ``quotes'' , instead of " "}
    \TODO{To emphasize a word, use {\em emphasize} and not ``quote''}
    \TODO{When using the characters \& \# \% \_  put a backslash before them so that they show up correctly}
    \TODO{Pasting and copying from the Web often results in non-ASCII characters to be used in your text, please remove them and replace accordingly. This is the case for quotes, dashes and all the other special characters.}
    \TODO{If you see a figure and not a figure in text you copied from a text that has the fi combined as a single character}

\subsection{Structural Issues}

    \TODO{Acknowledgement section missing}
    \TODO{Incorrect README file}
    \TODO{In case of a class and if you do a multi-author paper, you need to add an appendix describing who did what in the paper}
    \TODO{The paper has less than 2 pages of text, i.e. excluding images, tables and figures}
    \TODO{The paper has more than 6 pages of text, i.e. excluding images, tables and figures}
    \TODO{Do not artificially inflate your paper if you are below the page limit}

\subsection{Details about the Figures and Tables}

    \TODO{Capitalization errors in referring to captions, e.g. Figure 1, Table 2}
    \TODO{Do use {\em label} and {\em ref} to automatically create figure numbers}
    \TODO{Wrong placement of figure caption. They should be on the bottom of the figure}
    \TODO{Wrong placement of table caption. They should be on the top of the table}
    \TODO{Images submitted incorrectly. They should be in native format, e.g. .graffle, .pptx, .png, .jpg}
    \TODO{Do not submit eps images. Instead, convert them to PDF}

    \TODO{The image files must be in a single directory named "images"}
    \TODO{In case there is a powerpoint in the submission, the image must be exported as PDF}
    \TODO{Make the figures large enough so we can read the details. If needed make the figure over two columns}
    \TODO{Do not worry about the figure placement if they are at a different location than you think. Figures are allowed to float. For this class, you should place all figures at the end of the report.}
    \TODO{In case you copied a figure from another paper you need to ask for copyright permission. In case of a class paper, you must include a reference to the original in the caption}
    \TODO{Remove any figure that is not referred to explicitly in the text (As shown in Figure ..)}
    \TODO{Do not use textwidth as a parameter for includegraphics}
    \TODO{Figures should be reasonably sized and often you just need to
  add columnwidth} e.g. \begin{verbatim}/includegraphics[width=\columnwidth]{images/myimage.pdf}\end{verbatim}

re\end{document}
